\clearpage
\thispagestyle{empty}

\noindent{\large\bf\dadoTituloAbs}
\noindent{\large\dadoSubTituloAbs}

\normalsize
\begin{center}	
	\vspace*{0.5cm}
	\textbf{\textit{ABSTRACT}}
\end{center}

The process of experimentation in Software Engineering (ES) is fundamental to the life cycle of a software. It is possible to reduce major development efforts and mainly maintenance. The ES community has been discussing and evaluating how to improve the quality of the experiments, in order to increase the reliability of its results. As much as they have approached the quality of generally controlled experiments, there is as yet no evidence they are analyzing in specific contexts, such as Software Product Lines (LPS). In this case, there is still a lack of instrumentation and specific measurement of the quality of experiments in LPS. It is therefore necessary to provide a reliable and replicable body of knowledge in the context of LPS. Because of this importance, designing, executing and analyzing the results of an experiment in LPS becomes crucial to guarantee the quality of the experiments. In this sense we propose an ontology (SMartyOntology) for experiments in LPS, because it has hundreds of published experiments. The ontology is primarily designed based on defined guidelines and is designed using OWL language, supported by the Protégé environment for syntax checking and initial evaluation. The ontology was filled with more than 150 experiments in software product lines, assembled in a systematic mapping study. It is also the opportunity to investigate the elaboration of a recommendation system for experiments in LPS based on the information modeling structured by the proposed ontology. Therefore, this paper presents fundamental concepts for the elaboration of an ontology for experiments in LPS, and for the creation of a recommendation system for experiments in LPS. We believe that this ontology as well as the recommendation system can contribute to better document the essential elements of an experiment, thus promoting the replication, replication and reproducibility of the experiments. In which, it can bring quality to the experimental projects and results obtained through the recommended experiment. Ontologies and Recommendation Systems are well known in ES, it is believed to be possible to apply these theories to recommend experiments in LPS. We also present a feasibility evaluation of the proposed ontology. At the end we have a recommendation system that shows good results in recommendations for controlled experiments. It is also hoped with this project to contribute to the LPS community in order to improve the projects and execution of experiments, increasing the trust of the body of knowledge in order to transfer technology to industry\\

\noindent \textbf{\textit{Keywords:}} Controlled Software Experiment. Ontology. Quality of Experiments. Software Product Line. Systems of Recommendation in Software Engineering. Systems of Recommendation.

\pagebreak

 
