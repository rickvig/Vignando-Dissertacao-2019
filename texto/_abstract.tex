\clearpage
\thispagestyle{empty}

\noindent{\large\bf\dadoTituloAbs}
\noindent{\large\dadoSubTituloAbs}

\normalsize
\begin{center}	
	\vspace*{0.5cm}
	\textbf{\textit{ABSTRACT}}
\end{center}

The process of experimentation in Software Engineering (ES) has been fundamental for the software life cycle. With it it is possible to reduce major development efforts and especially maintenance. The higher education community has been discussing and evaluating how to improve the quality of higher education experiments in order to increase the reliability of their results. As much as the quality of experiments has been addressed, there is still a lack in specific contexts, such as Software Product Lines (LPS). Thus, planning, executing, and analyzing the results of an LPS experiment becomes crucial in supporting the evolution of LPS and providing a reliable and auditable body of knowledge. In this sense, this paper presents an ontology to support LPS experiments, OntoExper-SPL. The ontology was designed based on predefined guidelines, designed using the Ontology Web Language (OWL) text and supported by the Prot�g� environment. OntoExper-SPL was populated with over 200 LPS experiments. The ontology was evaluated based on a feasibility study with 17 LPS and ontology specialists. In addition, a Prototype Recommendation System (SR) capable of using ontology to make inferences about data from LPS experiments was implemented. Thus, it is believed that ontology can directly contribute to better documentation of LPS experiments, disseminate LPS experimentation knowledge, support the culture of experimentation in academia and industry, and improve software design and experiment execution by increasing the confidence of the body of knowledge to transfer technology to industry. \\

\noindent \textbf{\textit{Keywords:}} Experiment in SPL. Software Product Line. Ontology. Ontology in Software Engineering. Software Engineering Recommendation System. Recommendation Systems.

\pagebreak

 
