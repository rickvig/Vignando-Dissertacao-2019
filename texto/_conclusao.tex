\chapter{Conclusão}
\label{sec:conclusao}


\section{Considerações Iniciais}
\label{sec:concidaracoes_iniciais}

...

A realização de experimentos de SE no SPL traz informações mais relevantes para fornecer evidências de uma teoria para o mundo real. A capacidade de entender, estudar e replicar experimentos torna esse método ainda mais rico. Ao organizar os dados e informações sobre os experimentos de SE no SPL, acreditamos que essa tarefa é mais fácil. Usando as tecnologias adjacentes, tais como formalizar o conhecimento através de ontologias e inferências para gerar dados de informação personalizada através de um sistema de recomendação.

Neste trabalho, foi desenvolvida uma proposta para um modelo de ontologia (SMartyOntology) com o objetivo de organizar e estruturar o conhecimento adquirido sobre experimentos de SE em SPL. Este conhecimento foi previamente levantado em um trabalho de dissertação do nosso grupo de pesquisa, onde foram relatados mais de 200 artigos que relatam experimentos em SPL. Com este levantamento foi possível criar um modelo de ontologia para representar o conhecimento sobre experimentos em NPS, inserindo os dados dos artigos neste modelo, denominamos esta fase de inserção dos indivíduos na ontologia. Desta forma, estruturamos o TBox e o Abox para a SMartyOntology. Com a composição da ontologia podemos fazer uma breve avaliação sobre algumas armadilhas que o modelo de ontologia poderia ter, utilizamos a ferramenta OOPS! para esta avaliação. Também foi possível criar um exemplo simples de recomendação, fazendo a inferência das informações para a SMartyOntology.

A SMartyOntology se destaca por levar em conta um domínio específico de experimentos de SE. SPL são construídos através de um domínio de aplicação, semelhanças, avaliações do núcleo e variabilidades, que distingue um produto do outro dentro da família de produtos, todas essas características estão presentes na ontologia. No entanto, a SMartyOntology pode ser facilmente estendida a outros domínios de experimentos de SE, uma vez que todas as classes dentro da ontologia que lidam com SPL são subclasses de representações de experimentos em geral.

Os principais objetivos deste trabalho, estão relacionados a propor um sistema de recomendação que, possa gerar processos e diretrizes para realização de experimentos em LPS. Para foi realizado um estudo aprofundado nos conceitos de Sistemas de Recomendação em ES e modelos de Ontologias para representação dos dados levantados sobre a qualidade dos experimentos em LPS, encontrados no trabalho do grupo.

Portanto, acredita-se que após a realização do projeto de software, será possível desenvolver o sistema de recomendação para que os usuários que interagirem com este sistema, e possa receber recomendações de processos e diretrizes para suas pesquisas experimentais em LPS.

Como trabalho futuro, identificamos na avaliação vários pontos de melhoria na modelagem da ontologia, esses ajustes devem ser feitos para padronizar o modelo com o objetivo de possibilitar a extensão e divulgação do mesmo.


\section{Considerações Finais}
\label{sec:concidaracoes_finais}

...