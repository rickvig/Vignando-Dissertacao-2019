\clearpage
\thispagestyle{empty}

\noindent{\large\bf\dadoTitulo}
\noindent{\large\dadoSubTitulo}

\normalsize
\begin{center}	
	\vspace*{0.5cm}
	\textbf{RESUMO}
\end{center}

O processo de experimentação em Engenharia de Software (ES) é fundamental para ciclo de vida de um software. Com ele é possível reduzir grandes esforços de desenvolvimento e principalmente de manutenção. A comunidade de ES vem discutindo e avaliando como melhorar a qualidade dos experimentos, visando aumentar a confiabilidade dos seus resultados. Por mais que eles tem abordado a qualidade de experimentos controlados de forma geral, ainda não há evidencias que estão analisando em contextos específicos, como é o caso de Linhas de produto de Software (LPS). Neste caso, ainda existe uma falta de instrumentação e medição especifica da qualidade dos experimentos em LPS. Por isso torna-se necessário fornecer um corpo de conhecimento confiável e replicável no contexto de LPS. Devido essa importância, projetar, executar e analisar os resultados de um experimento em LPS torna-se crucial para garantir a qualidade dos mesmos. Neste sentido propomos uma ontologia (SMartyOntology) para experimentos em LPS, pois possui centenas de experimentos publicados. A ontologia é concebida principalmente com base em diretrizes definidas e é projetada usando linguagem OWL, suportada pelo ambiente Protégé para verificação de sintaxe e avaliação inicial. A ontologia foi preenchida com mais de 150 experimentos em linhas de produtos de software, reunidos em um estudo de mapeamento sistemático. Surge também a oportunidade de investigar a elaboração de um sistema de recomendação para experimentos em LPS se baseando em na modelagem de informação estruturada pela ontologia proposta. Portanto, este trabalho apresenta conceitos fundamentais para elaboração de uma ontologia voltada para experimentos em LPS, e para criação de um sistema de recomendação para experimentos em LPS. Acreditamos que essa ontologia  bem como o sistema de recomendação pode contribuir para documentar melhor os elementos essenciais de um experimento, promovendo assim a repetição, a replicação e a reprodutibilidade dos experimentos. Na qual, possa levar qualidade para os projetos experimentais e resultados obtidos por meio dos experimento recomendados. Ontologias e Sistemas de recomendação são bem conhecidos na ES, acredita-se ser possível aplicar essas teorias para recomendar experimentos em LPS. Apresentamos também uma avaliação de viabilidade da ontologia proposta. Ao final temos um sistema de recomendação que apresenta bons resultados em recomendações para experimentos controlado. Espera-se também com este projeto, contribuir com a comunidade de LPS no sentido de melhorar os projetos e execução de experimentos, aumentando a confiança do corpo de conhecimento visando a transferência de tecnologia para indústria.\\

\noindent \textbf{Palavras-chave:} Experimento Controlado de Software. Linha de Produto de Software. Ontologia. Qualidade de Experimentos. Sistemas de Recomendação em Engenharia de Software. Sistemas de Recomendação.

\pagebreak
