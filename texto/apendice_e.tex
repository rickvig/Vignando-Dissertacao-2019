\appendix{Ap�ndice D: C�digo Fonte dos Exemplos de Aplica��o de Recomenda��o Usando OntoExper-SPL }
\label{apendice:d}

\section{Arquivo \texttt{settings.py}}

\scriptsize
\begin{verbatim}
"""
Django settings for recsys project.

Generated by 'django-admin startproject' using Django 2.2.5.

For more information on this file, see
https://docs.djangoproject.com/en/2.2/topics/settings/

For the full list of settings and their values, see
https://docs.djangoproject.com/en/2.2/ref/settings/
"""

import os

# Build paths inside the project like this: os.path.join(BASE_DIR, ...)
BASE_DIR = os.path.dirname(os.path.dirname(os.path.abspath(__file__)))


# Quick-start development settings - unsuitable for production
# See https://docs.djangoproject.com/en/2.2/howto/deployment/checklist/

# SECURITY WARNING: keep the secret key used in production secret!
SECRET_KEY = 'kw5akwqdaq^&&k2uq=e8^pr!ij1le8$lva8htv3m*##6(khb98'

# SECURITY WARNING: don't run with debug turned on in production!
DEBUG = True

ALLOWED_HOSTS = []


# Application definition

INSTALLED_APPS = [
    'django.contrib.admin',
    'django.contrib.auth',
    'django.contrib.contenttypes',
    'django.contrib.sessions',
    'django.contrib.messages',
    'django.contrib.staticfiles',
    'rating',
]

MIDDLEWARE = [
    'django.middleware.security.SecurityMiddleware',
    'django.contrib.sessions.middleware.SessionMiddleware',
    'django.middleware.common.CommonMiddleware',
    'django.middleware.csrf.CsrfViewMiddleware',
    'django.contrib.auth.middleware.AuthenticationMiddleware',
    'django.contrib.messages.middleware.MessageMiddleware',
    'django.middleware.clickjacking.XFrameOptionsMiddleware',
]

ROOT_URLCONF = 'recsys.urls'

TEMPLATES = [
    {
        'BACKEND': 'django.template.backends.django.DjangoTemplates',
        'DIRS': ['templates'],
        'APP_DIRS': True,
        'OPTIONS': {
            'context_processors': [
                'django.template.context_processors.debug',
                'django.template.context_processors.request',
                'django.contrib.auth.context_processors.auth',
                'django.contrib.messages.context_processors.messages',
            ],
        },
    },
]

WSGI_APPLICATION = 'recsys.wsgi.application'


# Database
# https://docs.djangoproject.com/en/2.2/ref/settings/#databases

DATABASES = {
    'default': {
        'ENGINE': 'django.db.backends.sqlite3',
        'NAME': os.path.join(BASE_DIR, 'db.sqlite3'),
    }
}


# Password validation
# https://docs.djangoproject.com/en/2.2/ref/settings/#auth-password-validators

AUTH_PASSWORD_VALIDATORS = [
    {
        'NAME': 'django.contrib.auth.password_validation.UserAttributeSimilarityValidator',
    },
    {
        'NAME': 'django.contrib.auth.password_validation.MinimumLengthValidator',
    },
    {
        'NAME': 'django.contrib.auth.password_validation.CommonPasswordValidator',
    },
    {
        'NAME': 'django.contrib.auth.password_validation.NumericPasswordValidator',
    },
]


# Internationalization
# https://docs.djangoproject.com/en/2.2/topics/i18n/

LANGUAGE_CODE = 'en-us'

TIME_ZONE = 'UTC'

USE_I18N = True

USE_L10N = True

USE_TZ = True


# Static files (CSS, JavaScript, Images)
# https://docs.djangoproject.com/en/2.2/howto/static-files/

STATIC_URL = '/static/'


ONTOLOGY_FILE = 'ontology-populate-valid.owl'

\end{verbatim}


\section{Arquivo \texttt{views.py}}

\scriptsize
\begin{verbatim}
import os
from django.http import HttpResponse
from django.shortcuts import render
from owlready2 import *
from django.conf import settings
from rating.models import Rating
from .collaborativeFilter import CollaborativeFilter

FILTER_SOURCE_SPL = 'sources_spl'
FILTER_TYPE_EXP_SPL = 'type_experiments_spl'


def home(request):
    onto = get_ontology(os.path.join(settings.BASE_DIR, settings.ONTOLOGY_FILE)).load()

    rec_sys = CollaborativeFilter()
    recs = rec_sys.recomendation(request.user.username)
    print('rec-sys', recs)

    recomendations = []
    for rec in recs[:4]:
        id_exp = rec[1]
        exp = (onto.search(idExperiment = id_exp), onto.search(idExperimentSPL = id_exp))[len(onto.search(idExperiment = id_exp)) == 0]
        recomendations.append({'id': id_exp, 'title': exp[0].title[0]})
    
    data = {
        'recomendations': recomendations,
        'onto': onto,
        'type_experiments': _getTypesFrom(onto.TypeExperiment),
        'type_context_experiment': _getTypesFrom(onto.TypeContextExperiment),
        'type_context_selection': _getTypesFrom(onto.TypeContextSelection),
        'type_design_experiment': _getTypesFrom(onto.TypeDesignExperiment),
        'type_participants': _getTypesFrom(onto.TypeSelectionParticipantsObjects),
        FILTER_TYPE_EXP_SPL: _getTypesFrom(onto.TypeExperimentSPL),
        FILTER_SOURCE_SPL: ['AGM', 'SPLOT', 'SIMPLE', 'ArgoUML', 'Test'],
        # 'user': request.user.username
    }
    return render(request, 'index.html', data)


def filter_experiment(request):

    onto = get_ontology(os.path.join(settings.BASE_DIR, settings.ONTOLOGY_FILE)).load()

    filters = eval(request.GET.get('filters'))

    sourceSPL = ''
    typeExperimentsSpl = ''
    result = []

    for f in filters:
        if f.get('key') == FILTER_SOURCE_SPL:
            sourceSPL = f.get('val')

        if f.get('key') == FILTER_TYPE_EXP_SPL:
            typeExperimentsSpl = f.get('val')

    instance_typeExperimentSPL = list(filter(
        lambda el: el._name == typeExperimentsSpl, onto.TypeExperimentSPL.instances()))

    result = onto.search(is_a=onto.Abstract,
                            documentation=onto.search(is_a=onto.Documentation,
                                                    experiment=onto.search(wasTheSPLSourceUsedInformed="*%s*" % sourceSPL,
                                                                            typeExperimentSPL=instance_typeExperimentSPL, _case_sensitive=False)))

    experiments = []
    for r in result:
        experiments.append(
            {
                'id_experiment': r.idAbstract[0],
                'title': r.documentation[0].experiment[0].title[0],
                'abstract': r.abstractBackground[0],
            }
        )

    data = {'experiments': experiments}
    return render(request, 'experiments.html', data)


def rating_experiment(request):
    onto = get_ontology(os.path.join(settings.BASE_DIR, settings.ONTOLOGY_FILE)).load()
    id_experiment = int(request.GET.get('id'))
    rating = int(request.GET.get('rating'))
    username = request.user.username

    try:
        rating_persist = Rating.objects.get(user=username, id_experiment=id_experiment)
        rating_persist.rating = rating
        rating_persist.save()
        retorno = 'update'
    except Rating.DoesNotExist:
        rating_persist = Rating(user=username,
                    id_experiment=id_experiment,
                    rating=request.GET.get('rating'),
                    title_experiment=onto.search(is_a=onto.ExperimentSPL, idExperimentSPL=id_experiment)[0].title[0]
                    )
        rating_persist.save()
        retorno = 'create'
    
    return HttpResponse(retorno)


def _getTypesFrom(clazz):
    o_types = clazz.instances()
    types = map(lambda t: t._name, o_types)
    return list(types)
    
\end{verbatim}

\section{Arquivo \texttt{collaborativeFilter.py}}

\scriptsize
\begin{verbatim}
import os
from django.conf import settings
from owlready2 import *
from rating.models import Rating
import pandas as pd
import numpy as np


class CollaborativeFilter():

    def __init__(self):
        onto = get_ontology(os.path.join(settings.BASE_DIR,
                                         settings.ONTOLOGY_FILE)).load()

        dataset = {}
        for user, ratings in self.__get_ratings().items():
            experiments = {}

            for exp in onto.Experiment.instances():
                try:
                    idExperiment = exp.idExperiment[0]
                    if idExperiment in ratings.keys():
                        experiments[idExperiment] = ratings[idExperiment]
                    else:
                        experiments[idExperiment] = None
                except IndexError:
                    idExperiment = exp.idExperimentSPL[0]
                    if idExperiment in ratings.keys():
                        experiments[idExperiment] = ratings[idExperiment]
                    else:
                        experiments[idExperiment] = None

            if user in list(dataset.keys()):
                dataset[user].update(experiments)
            else:
                dataset[user] = experiments

        self.df = pd.DataFrame(dataset)
        # print(self.df)

    def __get_ratings(self):
        user_ratings = {}
        resultset = Rating.objects.all()

        for rs in resultset:
            if rs.user in list(user_ratings.keys()):
                user_ratings[rs.user].update({rs.id_experiment: rs.rating})
            else:
                user_ratings[rs.user] = {rs.id_experiment: rs.rating}

        return user_ratings

    def euclidean(self, usr1, usr2):
        self.df['sqr'] = self.df.loc[:, [usr1, usr2]].apply(
            lambda item: np.power(item[0] - item[1], 2), axis=1)

        return 1 / (1 + np.sqrt(self.df['sqr'].sum()))

    def recomendation(self, usr):
        df_usr = self.df.loc[:, [usr]]
        experiments = df_usr[df_usr[usr].isnull()].index

        totals = {}
        sum_similarity = {}
        for other in self.df.drop(columns=usr).columns:
            similarity = self.euclidean(usr, other)

            for item in experiments:
                if np.isnan(self.df[other][item]):
                    continue

                totals.setdefault(item, 0)
                totals[item] += self.df[other][item] * similarity

                sum_similarity.setdefault(item, 0)
                sum_similarity[item] += similarity

        rankings = [(total / sum_similarity[item], item)
                    for item, total in totals.items()]

        rankings.sort()
        rankings.reverse()

        return rankings

\end{verbatim}